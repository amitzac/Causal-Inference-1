\documentclass{beamer}

% xcolor and define colors -------------------------
\usepackage{xcolor}

% https://www.viget.com/articles/color-contrast/
\definecolor{purple}{HTML}{5601A4}
\definecolor{navy}{HTML}{0D3D56}
\definecolor{ruby}{HTML}{9a2515}
\definecolor{alice}{HTML}{107895}
\definecolor{daisy}{HTML}{EBC944}
\definecolor{coral}{HTML}{F26D21}
\definecolor{kelly}{HTML}{829356}
\definecolor{cranberry}{HTML}{E64173}
\definecolor{jet}{HTML}{131516}
\definecolor{asher}{HTML}{555F61}
\definecolor{slate}{HTML}{314F4F}

% Mixtape Sessions
\definecolor{picton-blue}{HTML}{00b7ff}
\definecolor{violet-red}{HTML}{ff3881}
\definecolor{sun}{HTML}{ffaf18}
\definecolor{electric-violet}{HTML}{871EFF}

% Main theme colors
\definecolor{accent}{HTML}{00b7ff}
\definecolor{accent2}{HTML}{871EFF}
\definecolor{gray100}{HTML}{f3f4f6}
\definecolor{gray800}{HTML}{1F292D}


% Beamer Options -------------------------------------

% Background
\setbeamercolor{background canvas}{bg = white}

% Change text margins
\setbeamersize{text margin left = 15pt, text margin right = 15pt} 

% \alert
\setbeamercolor{alerted text}{fg = accent2}

% Frame title
\setbeamercolor{frametitle}{bg = white, fg = jet}
\setbeamercolor{framesubtitle}{bg = white, fg = accent}
\setbeamerfont{framesubtitle}{size = \small, shape = \itshape}

% Block
\setbeamercolor{block title}{fg = white, bg = accent2}
\setbeamercolor{block body}{fg = gray800, bg = gray100}

% Title page
\setbeamercolor{title}{fg = gray800}
\setbeamercolor{subtitle}{fg = accent}

%% Custom \maketitle and \titlepage
\setbeamertemplate{title page}
{
    %\begin{centering}
        \vspace{20mm}
        {\Large \usebeamerfont{title}\usebeamercolor[fg]{title}\inserttitle}\\
        {\large \itshape \usebeamerfont{subtitle}\usebeamercolor[fg]{subtitle}\insertsubtitle}\\ \vspace{10mm}
        {\insertauthor}\\
        {\color{asher}\small{\insertdate}}\\
    %\end{centering}
}

% Table of Contents
\setbeamercolor{section in toc}{fg = accent!70!jet}
\setbeamercolor{subsection in toc}{fg = jet}

% Button 
\setbeamercolor{button}{bg = accent}

% Remove navigation symbols
\setbeamertemplate{navigation symbols}{}

% Table and Figure captions
\setbeamercolor{caption}{fg=jet!70!white}
\setbeamercolor{caption name}{fg=jet}
\setbeamerfont{caption name}{shape = \itshape}

% Bullet points

%% Fix left-margins
\settowidth{\leftmargini}{\usebeamertemplate{itemize item}}
\addtolength{\leftmargini}{\labelsep}

%% enumerate item color
\setbeamercolor{enumerate item}{fg = accent}
\setbeamerfont{enumerate item}{size = \small}
\setbeamertemplate{enumerate item}{\insertenumlabel.}

%% itemize
\setbeamercolor{itemize item}{fg = accent!70!white}
\setbeamerfont{itemize item}{size = \small}
\setbeamertemplate{itemize item}[circle]

%% right arrow for subitems
\setbeamercolor{itemize subitem}{fg = accent!60!white}
\setbeamerfont{itemize subitem}{size = \small}
\setbeamertemplate{itemize subitem}{$\rightarrow$}

\setbeamertemplate{itemize subsubitem}[square]
\setbeamercolor{itemize subsubitem}{fg = jet}
\setbeamerfont{itemize subsubitem}{size = \small}

% Special characters

\usepackage{collectbox}

\makeatletter
\newcommand{\mybox}{%
    \collectbox{%
        \setlength{\fboxsep}{1pt}%
        \fbox{\BOXCONTENT}%
    }%
}
\makeatother






% Links ----------------------------------------------

\usepackage{hyperref}
\hypersetup{
  colorlinks = true,
  linkcolor = accent2,
  filecolor = accent2,
  urlcolor = accent2,
  citecolor = accent2,
}


% Line spacing --------------------------------------
\usepackage{setspace}
\setstretch{1.2}


% \begin{columns} -----------------------------------
\usepackage{multicol}


% Fonts ---------------------------------------------
% Beamer Option to use custom fonts
\usefonttheme{professionalfonts}

% \usepackage[utopia, smallerops, varg]{newtxmath}
% \usepackage{utopia}
\usepackage[sfdefault,light]{roboto}

% Small adjustments to text kerning
\usepackage{microtype}



% Remove annoying over-full box warnings -----------
\vfuzz2pt 
\hfuzz2pt


% Table of Contents with Sections
\setbeamerfont{myTOC}{series=\bfseries, size=\Large}
\AtBeginSection[]{
        \frame{
            \frametitle{Roadmap}
            \tableofcontents[current]   
        }
    }


% Tables -------------------------------------------
% Tables too big
% \begin{adjustbox}{width = 1.2\textwidth, center}
\usepackage{adjustbox}
\usepackage{array}
\usepackage{threeparttable, booktabs, adjustbox}
    
% Fix \input with tables
% \input fails when \\ is at end of external .tex file
\makeatletter
\let\input\@@input
\makeatother

% Tables too narrow
% \begin{tabularx}{\linewidth}{cols}
% col-types: X - center, L - left, R -right
% Relative scale: >{\hsize=.8\hsize}X/L/R
\usepackage{tabularx}
\newcolumntype{L}{>{\raggedright\arraybackslash}X}
\newcolumntype{R}{>{\raggedleft\arraybackslash}X}
\newcolumntype{C}{>{\centering\arraybackslash}X}

% Figures

% \imageframe{img_name} -----------------------------
% from https://github.com/mattjetwell/cousteau
\newcommand{\imageframe}[1]{%
    \begin{frame}[plain]
        \begin{tikzpicture}[remember picture, overlay]
            \node[at = (current page.center), xshift = 0cm] (cover) {%
                \includegraphics[keepaspectratio, width=\paperwidth, height=\paperheight]{#1}
            };
        \end{tikzpicture}
    \end{frame}%
}

% subfigures
\usepackage{subfigure}


% Highlight slide -----------------------------------
% \begin{transitionframe} Text \end{transitionframe}
% from paulgp's beamer tips
\newenvironment{transitionframe}{
    \setbeamercolor{background canvas}{bg=accent!40!black}
    \begin{frame}\color{accent!10!white}\LARGE\centering
}{
    \end{frame}
}


% Table Highlighting --------------------------------
% Create top-left and bottom-right markets in tabular cells with a unique matching id and these commands will outline those cells
\usepackage[beamer,customcolors]{hf-tikz}
\usetikzlibrary{calc}
\usetikzlibrary{fit,shapes.misc}

% To set the hypothesis highlighting boxes red.
\newcommand\marktopleft[1]{%
    \tikz[overlay,remember picture] 
        \node (marker-#1-a) at (0,1.5ex) {};%
}
\newcommand\markbottomright[1]{%
    \tikz[overlay,remember picture] 
        \node (marker-#1-b) at (0,0) {};%
    \tikz[accent!80!jet, ultra thick, overlay, remember picture, inner sep=4pt]
        \node[draw, rectangle, fit=(marker-#1-a.center) (marker-#1-b.center)] {};%
}

\usepackage{breqn} % Breaks lines

\usepackage{amsmath}
\usepackage{mathtools}

\usepackage{pdfpages} % \includepdf

\usepackage{listings} % R code
\usepackage{verbatim} % verbatim

% Video stuff
\usepackage{media9}

% packages for bibs and cites
\usepackage{natbib}
\usepackage{har2nat}
\newcommand{\possessivecite}[1]{\citeauthor{#1}'s \citeyearpar{#1}}
\usepackage{breakcites}
\usepackage{alltt}

% tikz
\usepackage{tikz}
\usepackage{pgfplots}
\usetikzlibrary{calc, positioning, decorations.pathreplacing, arrows.meta, intersections}
\pgfdeclarelayer{bg}
\pgfdeclarelayer{back}
\pgfdeclarelayer{fg}
\pgfsetlayers{bg,main,fg,back}
\usetikzlibrary{shapes,arrows}

% Setup math operators
\DeclareMathOperator{\E}{E} \DeclareMathOperator{\tr}{tr} \DeclareMathOperator{\se}{se} \DeclareMathOperator{\I}{I} \DeclareMathOperator{\sign}{sign} \DeclareMathOperator{\supp}{supp} \DeclareMathOperator{\plim}{plim}
\DeclareMathOperator*{\dlim}{\mathnormal{d}\mkern2mu-lim}
\newcommand\independent{\protect\mathpalette{\protect\independenT}{\perp}}
   \def\independenT#1#2{\mathrel{\rlap{$#1#2$}\mkern2mu{#1#2}}}
\newcommand*\colvec[1]{\begin{pmatrix}#1\end{pmatrix}}

\newcommand{\myurlshort}[2]{\href{#1}{\textcolor{gray}{\textsf{#2}}}}


\begin{document}

\imageframe{./lecture_includes/mixtape_ci_cover.png}


% ---- Content ----










\section{Foundational ideas}


\begin{frame}{What is Causality?}

\begin{itemize}
\item Causality is metaphysics (what is?); causal inference is epistemology (what is knowledge?) IOW, what makes a causal belief a ``warranted belief''?
\item Causality and causal inference have common sources (e.g., Aristotle, Hume, Mill, Lewis) but we will focus primarily on the experimental design tradition
\item We use the potential outcomes framework of causal inference to build and discuss methods that estimate unbiased and meaningful \textbf{average} causal parameters as defined by that framework
\end{itemize}

\end{frame}


\begin{frame}{Causal Inference vs Prediction}
  \centering
  \includegraphics[scale=0.5,height=6.5cm, width=10cm]{./lecture_includes/prediction_causality.png}
\end{frame}

\begin{frame}{Causal Inference vs Prediction}

  \begin{columns}
    \column{0.48\linewidth}
    \centering
    \textbf{Traditional prediction}
    \begin{itemize}
      \item Traditional prediction seeks to detect patterns in data and fit functional relationships between variables with a high degree of accuracy
      \item ``Does this person have heart disease?'', ``How many books will I sell?''
      \item It is not predictions of what effect a choice will have, though
    \end{itemize}
    \column{0.48\linewidth}
    \centering
    \textbf{Causal inference}
    \begin{itemize}
      \item Causal inference is also a type of prediction, but it's a prediction of a \emph{counterfactual} associated with a particular \emph{choice taken}
      \item Causal inference takes that predicted (or imputed) counterfactual and constructs a causal effect that we hope tells us about a future in the event of a similar choice taken
    \end{itemize}
  \end{columns}
\end{frame}



\begin{frame}{Naive causal inference}

  \begin{itemize}
    \item Aliens estimate a model showing a systematic correlation between COVID deaths and ventilators
    \item They conclude doctors are killing patients with ventilators so they come to earth to liberate the patients, but it only makes things worse
    \item Their error was they confused correlation with causality, but deeper than that, they didn't understand how the world worked
    \item \emph{We are the aliens in our research}
  \end{itemize}

\end{frame}

\begin{frame}{\#1: Correlation and causality are different concepts}

  Causal is one unit, correlation is many units
  \begin{itemize}
    \item Causal question: ``If a doctor puts a patient on a ventilator (D), will her covid symptoms (Y) improve?''
    \item Correlation question:  $$\frac{Cov(D,Y)}{\sqrt{Var_D}\sqrt{{Var_Y}}}$$
    \item Error extends to predictive modeling that isn't based on causal frameworks
  \end{itemize}

\end{frame}



\begin{frame}{\#2: Coming first may not mean causality!}

  \begin{itemize}
    \item Every morning the rooster crows and then the sun rises
    \item Did the rooster cause the sun to rise? Or did the sun cause the rooster to crow?
    \item What if cat killed the rooster?
    \item \emph{Post hoc ergo propter hoc}: ``after this, therefore, because of this''
  \end{itemize}

\end{frame}

\begin{frame}{\#3: Causality may mask correlations!}

  \begin{figure}
    \centering
    \includegraphics[scale=0.04]{./lecture_includes/scottboat.jpg}
  \end{figure}

\end{frame}






\begin{frame}{Modeling is Not the First Step}

Most of us simply estimate models and cross our fingers that that coefficient is causal, but is it? When is it?  Why is it? And which causal effect is it?  And when is it reasonable to believe it? 

\bigskip

We have to introduce concepts and notation first otherwise we will extend the correlation fallacy


\end{frame}



\begin{frame}{Definition and Identification Come First}

\begin{enumerate}
\item Turn the research question (``what is the causal effect of an advertising campaign on sales?'') into a specific aggregate causal parameter
\item Describe the narrow set of beliefs that make that parameter obtainable with data
\item Build a model that uses the data and the beliefs to estimate the causal parameter?
\end{enumerate}

\bigskip

Most of us skip (1) and many skip (2) and go straight to (3) but hopefully today I'll convince you that that's how errors are introduced, even after one understands that causal inference is not merely correlational

\end{frame}

\section{Brief (and selected) history of causal inference}

\begin{frame}{Three New Ideas}

\begin{enumerate}
\item \textbf{Counterfactual}: Philosophers come to it first and its central role in causal inference makes causality \emph{unknowable} that the project is nearly derailed
\item \textbf{Treatment assignment mechanism}: Neyman and Fisher solve the counterfactual problem in statistics and lay the foundation of the modern randomized controlled trial (RCT) with their focus on the selection process
\item \textbf{No One Causal Effect}: There is no such thing as ``the causal effect''; there's many and your first step is to pick a parameter (not as easy as it sounds)
\end{enumerate}


\end{frame}



\begin{frame}{Modern Philosophers Introduce Counterfactual Comparisons}

\begin{quote}
    ``If a person eats of a particular dish, and dies in consequence, that is, would not have died if he had not eaten it, people would be apt to say that eating of that dish was the source of his death.'' -- John Stuart Mill (19th century moral philosopher and economist)
\end{quote}

\bigskip
  
    \begin{quote}
    ``Causation is something that makes a difference, and the difference it makes must be a difference from what would have happened without it.'' -- David Lewis (20th century philosopher)
\end{quote}

\end{frame}

  
\begin{frame}{Counterfactuals Almost Derailed Causal Inference}



Mill's counterfactuals were immensely valuable for the clarity of the definition as well as its intuitive validity of causality, but it came at a huge price 

\bigskip

If I have to know what would have happened had I not eaten the dish, but I did eat the dish, then isn't it actually impossible to know the causal effect of eating the dish?

\bigskip

Statisticians surprisingly resolve this tension in the early 20th century with the introduction of notation and the principles of treatment assignment


\end{frame}


\begin{frame}{Statistical origins}

\begin{quote}
``Yet, although the seeds of the idea that [causal effects are comparisons of potential outcomes] can be traced back at least to the 18th century [most likely he means David Hume], the formal notation for potential outcomes was not introduced until 1923 by Neyman.'' -- Don Rubin (1990)
\end{quote}

\end{frame}


\begin{frame}{Jerzy Neyman's Notation}

\begin{itemize}
\item Jerzy Neyman's 1923 masters thesis describes a field experiment with differing plots of land (imagine hundreds of square gardens) and many different ``varieties'' of fertilizer that farmers could apply to the land
\item ``$U_{ik}$ is the yield of the $i$th variety on the $k$th plot...'' (Neyman 1923)
\item He calls $U_{ik}$ ``potential yield'', as opposed to the realized yield because $i$ (the fertilizer type) described all possible fertilizers that could be assigned to each $k$ square garden
\item Though only one fertilizer will be assigned to the land, many possible fertilizer assignments were possible beforehand, each with their own outcome
\end{itemize}

\end{frame}

\begin{frame}{Jerzy Neyman's Notation}

\begin{itemize}

\item For each fertilizer there is an associated ``potential yield'' that he collapses into $U$ which he considers to be ``a priori fixed but unknown'' (Rubin 1990)
\item Farmers draw fertilizer from an urn, like a bingo ball from a bingo ball machine, with replacement and apply it to each square garden
\item Fertilizer assignment moves us from ``all possible outcomes'' to ``realized outcome'' terminology
\item Neyman's urn model was a classic thought experiment, but it was also stochastically identical to the completely randomized experiment
\item His arch-rival, Ronald Fisher, realizes this and publishes a book two years later calling for \emph{randomizization} as the basis for causal inference
\end{itemize}

\end{frame}

\begin{frame}{Treatment assignment mechanism}

\begin{quote}

``Before the 20th century, there appears to have been only limited awareness of the concept of the assignment mechanism.  Although by the 1930s, randomized experiments were firmly established in some areas of scientific investigation, notably in agricultural experiments, there was no formal statement for a general assignment mechanism and, moreover, not even formal arguments in favor of randomization until Fisher (1925).'' (Imbens and Rubin 2015)

\end{quote}

\end{frame}

\begin{frame}{Progress is made and progress is not made}

\begin{itemize}

\item Econometrics traditionally modeled causality in terms of realized outcomes until recently (with some exceptions)
\item We need to make a distinction between now the idea of data (``realized outcomes'') and these hypothetical concepts represented by Neyman's notation (``potential outcomes'')
\item Listen to Guido Imbens describe the transition towards modeling causality in terms of ``realized outcomes''

\end{itemize}

\bigskip

\url{https://www.youtube.com/watch?v=drGkRy53bB4}

\end{frame}



\begin{frame}{Potential outcomes notation}

Let the treatment be a binary variable: $$D_{i,t} =\begin{cases} 1 \text{ if placed on ventilator at time $t$} \\ 0 \text{ if not placed on ventilator at time $t$} \end{cases}$$where $i$ indexes an individual observation, such as a person
\end{frame}

\begin{frame}{Potential outcomes notation}

Potential outcomes: $$Y_{i,t}^j =\begin{cases} 1 \text{ health if placed on ventilator at time $t$} \\ 0 \text{ health if not placed on ventilator at time $t$} \end{cases}$$where $j$ indexes a potential treatment status for the same $i$ person at the same $t$ point in time
\end{frame}


\begin{frame}{Realized vs potential outcomes}

  \begin{itemize}
    \item Potential outcome $Y^1$ refers to the ``a priori fixed but unknown'' outcomes associated with different possible treatment assignments
    \item Realized outcome $Y$ refers to the ``posterior and known'' outcome associated with a specific treatment assignment
    \item Potential outcomes become realized outcomes through treatment assignment generated by an assignment mechanism like randomization or rationality
  \end{itemize}
\end{frame}


\begin{frame}{Models vs Treatment Assignment}

\begin{itemize}
    \item Treatment assignment \emph{mechanism} drives the entire effort to identify causal effects as some make it easy and some make it potentially \emph{impossible}
	\item Put another way, the same model can be unbiased and biased depending on the treatment assignment and be utterly detectable otherwise
	\item Means modeling does not come first -- it comes last
	\end{itemize}
\end{frame}

\begin{frame}{Important definitions}

    \begin{block}{Definition 1: Individual treatment effect}
      The individual treatment effect,  $\delta_i$, associated with a ventilator is equal to $Y_i^1-Y_i^0$.
    \end{block}
\end{frame}


\begin{frame}{Important definitions}


    \begin{block}{Definition 2: Switching equation}
      An individual's realized health outcome, $Y_i$, is determined by treatment assignment, $D_i$ which selects one of the potential outcomes:
      \begin{eqnarray*}
        Y_i& = D_iY^1_i+(1-D_i)Y^0_i& \\
        Y_i& = \begin{cases}
          Y^1_i\text{ if }D_i=1 \\
          Y^0_i\text{ if }D_i=0
        \end{cases}
      \end{eqnarray*}
    \end{block}
    
    Not the same as treatment assignment mechanism.  Treatment assignment mechanism describes how $D$ was assigned, not whether it was assigned.

\end{frame}


\begin{frame}{Missing data problem}


    \begin{block}{Definition 3: Fundamental problem of causal inference}
      If you need both potential outcomes to know causality with certainty, then since it is impossible to observe both $Y_i^1$ and $Y_i^0$ for the same individual, $\delta_i$, is \emph{unknowable}.
    \end{block}

This is my reason from saying Mill's counterfactual framework derailed the quest for causal effects given counterfactuals are fictional
    
\end{frame}

\begin{frame}{Missing data problem}


    
      \begin{itemize}
    \item Fundamental problem of causal inference is deep and impossible to overcome -- not even with more data (you will always with more data be missing one of the potential outcomes)
    \item Causal inference is a missing data problem 
    \item All of causal inference involves imputing missing counterfactuals and not all imputations are equal
  \end{itemize}

    
\end{frame}




\begin{frame}{Average Treatment Effects}

  \begin{block}{Definition 4: Average treatment effect (ATE)}
    The average treatment effect is the population average of all $i$ individual treatment effects
    \begin{eqnarray*}
      E[\delta_i]&=&E[Y_i^1-Y_i^0]\\
      &=&E[Y^1_i] - E[Y^0_i]
    \end{eqnarray*}
  \end{block}

  \bigskip

Aggregate parameters based on individual treatment effects are \emph{summaries} of individual treatment effects

\bigskip

  Cannot be calculated because $Y^1_i$ and $Y^0_i$ do not exist \emph{for the same unit i} due to switching equation

\end{frame}



\begin{frame}{Conditional Average Treatment Effects}


  \begin{block}{Definition 5: Average Treatment Effect on the Treated (ATT)}
    The average treatment effect on the treatment group is equal to the average treatment effect conditional on being a treatment group member:
    \begin{eqnarray*}
      E[\delta|D=1]&=&E[Y^1-Y^0|D=1] \nonumber \\
      &=&E[Y^1|D=1]-E[Y^0|D=1]
    \end{eqnarray*}
  \end{block}
  Cannot be calculated because $Y^1_i$ and $Y^0_i$ do not exist \emph{for the same unit i} due to switching equation. 


\end{frame}



\begin{frame}{Conditional Average Treatment Effects}

  \begin{block}{Definition 6: Average Treatment Effect on the Untreated (ATU)}
    The average treatment effect on the untreated group is equal to the average treatment effect conditional on being untreated:
    \begin{eqnarray*}
      E[\delta|D=0]&=&E[Y^1-Y^0|D=0] \nonumber \\
      &=&E[Y^1|D=0]-E[Y^0|D=0]
    \end{eqnarray*}
  \end{block}
  Cannot be calculated because $Y^1_i$ and $Y^0_i$ do not exist \emph{for the same unit i} due to switching equation

\end{frame}


\begin{frame}{Average Treatment Effects are Simple Summaries}

  \begin{itemize}
	\item Notice how in all three of these, all we did was take the defined treatment effect at the individual and aggregate
	\item Because aggregate causal parameters are \emph{summaries} of individual treatment effects, each of which cannot be calculated, the aggregates cannot be calculated either
	\item Missing data in this context isn't missing your car keys -- it's missing unicorns and fire breathing dragons (fictional vs real data)
	\item While we cannot measure average causal effects, we can estimate them, but only in situations and we review one -- randomization
  \end{itemize}

\end{frame}







\begin{frame}{Simple Comparisons}



  \begin{block}{Definition 7: Simple difference in mean outcomes (SDO)}
    A simple difference in mean outcomes (SDO) can be approximated by comparing the sample average outcome for the treatment group ($D=1$) with a comparison group ($D=0$)
    
    \begin{eqnarray*}
      SDO &=& E[Y^1 | D=1] - E[Y^0 | D=0]
    \end{eqnarray*}
  \end{block}
  \bigskip

SDO is not a causal parameter because it's comparing $Y^1$ and $Y^0$ for different units, not the same units, so what is it measuring? 

\end{frame}


\begin{frame}{Decomposition of the SDO}

  \begin{block}{Decomposition of the SDO}
    The SDO is made up of three things:
    \begin{eqnarray*}
      E[Y^1 | D=1] - E[Y^0 | D=0]&=& ATE\nonumber \\
      &&+ E[Y^0|D=1] - E[Y^0|D=0] \nonumber \\
      && + (1-\pi)(ATT - ATU)
    \end{eqnarray*}
  \end{block}

\bigskip

where $\pi$ is the share of units in the treatment group
\end{frame}


\begin{frame}{Begin with ATE definition}

  \begin{block}{Law of iterated expectations}
    \begin{eqnarray*}
      \text{ATE}&=&E[Y^1]-E[Y^0]  \\
      &=& \{\pi E[Y^1 | D=1] + (1-\pi)E[Y^1 | D=0]\}  \\
      & & - \{\pi E[Y^0|D=1] + (1-\pi) E[Y^0 | D=0]\}
    \end{eqnarray*}
  \end{block}

\bigskip

ATE is sum of four conditional expectations (can also be rearranged as a weighted average of the ATT and the ATU)


\end{frame}

\begin{frame}{Change notation}



  \begin{block}{Substitute letters for expectations}
    \begin{eqnarray*}
      E[Y^1|D=1] &=& a  \\
      E[Y^1|D=0] &=& b  \\
      E[Y^0|D=1] &=& c  \\
      E[Y^0|D=0] &=& d  \\
      \text{ATE} &=& e
    \end{eqnarray*}
  \end{block}
  



\end{frame}

\begin{frame}{Rewrite ATE definition}


  \begin{block}{Rewrite ATE}
    \begin{eqnarray*}
      e&=&\{\pi{a} + (1-\pi)b\} - \{\pi{c} + (1-\pi)d\}
    \end{eqnarray*}
  \end{block}

\end{frame}




\begin{frame}[plain]

  \begin{block}{Simple manipulation of ATE definition}
    \begin{eqnarray*}
      e&=&\{\pi{a} + (1-\pi)b\} - \{\pi{c} + (1-\pi)d\}  \\
      e&=&\pi{a} + b - \pi{b} - \pi{c} - d + \pi{d}  \\
      e&=&\pi{a} + b - \pi{b} - \pi{c} - d + \pi{d} + (\textbf{a} - \textbf{a}) + (\textbf{c} - \textbf{c}) + (\textbf{d} - \textbf{d})  \\
      0&=&e-\pi{a} - b + \pi{b} + \pi{c} + d - \pi{d} - \textbf{a} + \textbf{a} - \textbf{c} + \textbf{c} - \textbf{d} + \textbf{d}  \\
      \textbf{a}-\textbf{d}&=&e-\pi{a} - b + \pi{b} + \pi{c} + d - \pi{d}  + \textbf{a} - \textbf{c} + \textbf{c} - \textbf{d}  \\
      \textbf{a}-\textbf{d}&=&e  + (\textbf{c} - \textbf{d}) + \textbf{a}-\pi{a} - b + \pi{b} - \textbf{c} + \pi{c} + d - \pi{d} \\
      \textbf{a}-\textbf{d}&=&e  + (\textbf{c} - \textbf{d}) + (1-\pi)a -(1-\pi)b + (1-\pi)d - (1-\pi)c  \\
      \textbf{a}-\textbf{d}&=&e  + (\textbf{c} - \textbf{d}) + (1-\pi)(a-c) -(1-\pi)(b-d)
    \end{eqnarray*}
  \end{block}


\end{frame}

\begin{frame}[shrink=20,plain]
  \begin{block}{Carry forward from previous slide}
    \begin{eqnarray*}
      \textbf{a}-\textbf{d}&=&e  + (\textbf{c} - \textbf{d}) + (1-\pi)(a-c) -(1-\pi)(b-d)
    \end{eqnarray*}
  \end{block}

  \begin{block}{Replace letters with original terms }
    \begin{eqnarray*}
      E[Y^1|D=1] - E[Y^0|D=0] &=& \alert{\text{ATE}}  \\
      &&+ (\alert{E[Y^0|D=1]} - E[Y^0|D=0])  \\
      && + (1-\pi)( \underbrace{\{E[Y^1|D=1] - \alert{E[Y^0|D=1]}\}}_{ \mathclap{\text{ATT}}}  \\
      && - (1-\pi)( \underbrace{\{\alert{E[Y^1|D=0]} - {E[Y^0|D=0]}\}}_{ \mathclap{\text{ATU}}}  \\
    \end{eqnarray*}
  \end{block}
  
$\alert{\text{Purple terms}}$ are explicitly missing counterfactuals 
  
\end{frame}

\begin{frame}{Decomposition of the SDO}

  \begin{block}{Decomposition of the SDO}
    \begin{eqnarray*}
      E[Y^1 | D=1] - E[Y^0 | D=0]  &=& \alert{ATE} \\
      &&+ (\alert{E[Y^0|D=1]} - E[Y^0|D=0])  \\
      && + (1-\pi)(\alert{ATT - ATU})
    \end{eqnarray*}
  \end{block}
  
  \bigskip
  
  Note: this is a \emph{rewritten} formula for the definition of the ATE and so it is an identity and thus \emph{always} true.  Also, we started with $\pi$ but in the end we weight by $1-\pi$.
\end{frame}


\begin{frame}[plain]

  \begin{block}{Estimate SDO with sample averages}
    \begin{eqnarray*}
      \underbrace{E_N[Y_i | D_i=1] - E_N[Y_i | D_i=0]}_{ \mathclap{\text{Estimate of SDO}}}&=& \underbrace{E[Y^1] - E[Y^0]}_{ \mathclap{\text{Average Treatment Effect}}} \\
      &&+ \underbrace{\alert{E[Y^0|D=1]} - E[Y^0 | D=0]}_{ \mathclap{\text{Selection bias}}}  \\
      && + \underbrace{(1-\pi)(ATT - ATU)}_{ \mathclap{\text{Heterogenous treatment effect bias}}}
    \end{eqnarray*}
  \end{block}

\bigskip

Using the switching equation and sample averages, we can calculate $E_N[Y|D=1] \to E[Y^1 | D=1]$, $E_N[Y|D=0] \to E[Y^0|D=0]$ and $(1-\pi)$ is the share of the population in the control group.

\end{frame}

\begin{frame}{Labor by design}

\begin{itemize}

	\item Hull, Kolesar and Walters (2022) tells the economic history of causal inference which is rooted in the Princeton Industrial Relations Section in the 70s and 80s responding to an ``empirical crisis''
	\item The core problem in causal inference is \emph{always} selection bias due to the fact that we cannot observe each units' counterfactual
	\item Historically two ways that this selection bias was addressed: modeling it directly (Heckman, and others) and by design
	\item Design is the experimental design rooted in Fisher and Neyman's work in the 20s, but the Section took it as a guiding principle even outside of the lab

\end{itemize}

\end{frame}




\begin{frame}{Selection bias}

\begin{itemize}
\item Selection bias in surveys has to do with non-representative samples; that is not its meaning in causal inference
\item Selection bias in causal inference is when the average potential outcome for treatment group and control group are different in a large sample
\item Source of the bias is the treatment assignment \emph{mechanism}
\item Examples of a treatment assignment mechanism: 
	\begin{enumerate}
	\item randomization (i.e., taking the medicine because a coin flip told you to take it) versus 
	\item sorting on one or both potential outcome (i.e., taking the medicine because it'll help you) which I call the ``Perfect Doctor''
	\end{enumerate}

\end{itemize}

\end{frame}

\begin{frame}{Bias \#1: Selection bias}

  \begin{itemize}
    \item Look very closely at the selection bias terms on their left and right hand sides $$\alert{E[Y^0|D=1]} \neq E[Y^0 |D=0]$$
	\item Interpretation
		\begin{enumerate}
	    \item When you say you chose something because you'd be unhappy if you didn't, that's selection on $Y^0$
    \item When you say you chose something because you'd be happy if you did it, that's selection on $Y^1$
    \item When you say you chose something because you'll be more happy if you do it than if you didn't do it, that's selection on $Y^1-Y^0$
    	\end{enumerate}
	\item All three of those create selection bias because all three assign treatments to units mechanically based on potential outcomes, which is always the source of selection bias
      \end{itemize}

\end{frame}



\begin{frame}{Humans cause selection bias, not statistical model}

\begin{itemize}
\item Paradox: the better our programs get, the better that humans get at doing things, the more interventions are targeting units based on treatment gains, the less reliable correlations are
\item Paradox: the less efficient our programs are, the more erratically human are at assigning interventions, the more reliable correlations are
\item The former creates strong selection bias; the latter is more like a randomized experiment
\end{itemize}

\end{frame}






\begin{frame}{Illustrating selection bias with spreadsheets}
\begin{itemize}
\item Eliminating selection bias requires understanding the selection mechanism -- why did units end up treated but not others?
\item Illustrate with Perfect Doctor exercise -- doctor knows each person's treatment effects, despite counterfactuals, and assigns treatment based on whether gains are positive or not
\item Illustrate decomposition using numerical example (pass out handout)
\end{itemize}
\end{frame}





\section{Independence and Selection Bias}

\begin{frame}{Summarizing the goals of causal inference}

  Our goal in causal inference is to estimate aggregate causal parameters with data using treatment assignment mechanisms that plausibly eliminate selection bias

  \bigskip

Depending on the treatment assignment mechanism, certain procedures are allowed and others are prohibited

  \bigskip

  Let's look what happens in an RCT \emph{and why} this addresses selection bias term $\alert{E[Y^0|D=1]}$ and $E[Y^0|D=0]$ to see why Fisher (1925) recommended it

\end{frame}


\begin{frame}{Independence}


  \begin{block}{Independence assumption}
    Treatment is assigned to a population independent of that population's potential outcomes  $$(Y^0,Y^1)\independent{D}$$
  \end{block}
  This is random or quasi-random assignment and ensures mean potential outcomes for the treatment group and control group are the same.  Also ensures other variables are distributed the same for a large sample.
  \begin{eqnarray*}
    \alert{E[Y^0|D=1]} &=& E[Y^0 | D=0] \\
    E[Y^1|D=1] &=& \alert{E[Y^1 | D=0]}
  \end{eqnarray*}
\end{frame}

\begin{frame}{Random Assignment Solves the Selection Problem}

  \begin{eqnarray*}
    \underbrace{E_N[y_i | d_i=1] - E_N[y_i | d_i=0]}_{ \mathclap{\text{SDO}}}&=& \underbrace{E[Y^1] - E[Y^0]}_{ \mathclap{\text{Average Treatment Effect}}} \\
    &&+ \underbrace{E[Y^0|D=1] - E[Y^0 | D=0]}_{ \mathclap{\text{Selection bias}}}  \\
    && + \underbrace{(1-\pi)(ATT - ATU)}_{ \mathclap{\text{Heterogenous treatment effect bias}}}
  \end{eqnarray*}


  \begin{itemize}
    \item If treatment is independent of potential outcomes, then swap out equations and \textbf{selection bias} zeroes out:
          \begin{eqnarray*}
            E[Y^0 | D=1] - E[Y^0 | D=0] &=& 0
          \end{eqnarray*}
  \end{itemize}

\end{frame}

\begin{frame}[shrink=20,plain]
  \begin{center}
    \textbf{Random Assignment Solves the Heterogenous Treatment Effects}
  \end{center}

  \begin{itemize}
    \item How does randomization affect heterogeneity treatment effects bias from the third line?  Rewrite definitions for ATT and ATU:\begin{eqnarray*}
            \text{ATT} = E[Y^1 | D=1] - E[Y^0 | D=1] \\
            \text{ATU} = E[Y^1 | D=0] - E[Y^0 | D=0] \\
          \end{eqnarray*}
    \item Rewrite the third row bias after $1-\pi$:\begin{eqnarray*}
            ATT - ATU &=& \textbf{E[Y$^1$ $|$ D=1]} - E[Y^0 | D=1] \\
            && - \textbf{E[Y$^1$ $|$ D=0]} + E[Y^0 | D=0] \\
            &=& 0
          \end{eqnarray*}
    \item If treatment is independent of potential outcomes, then:\begin{eqnarray*}
            E_N[y_i | d_i=1] - E_N[y_i | d_i=0]  &=& E[Y^1] - E[Y^0] \\
            SDO &=& ATE
          \end{eqnarray*}
  \end{itemize}
\end{frame}






\begin{frame}[plain]

  \begin{block}{Identification with Randomization}
    \begin{eqnarray*}
      \underbrace{E_N[Y_i | D_i=1] - E_N[Y_i | D_i=0]}_{ \mathclap{\text{Estimate of SDO}}}&=& \underbrace{E[Y^1] - E[Y^0]}_{ \mathclap{\text{Average Treatment Effect}}} \\
    &&+ \underbrace{0}_{ \mathclap{\text{Selection bias}}}  \\
    && + \underbrace{0}_{ \mathclap{\text{Heterogenous treatment effect bias}}}
    \end{eqnarray*}
  \end{block}
  
  SDO is unbiased estimate of ATE with randomized treatment assignment because it sets selection bias to zero and $ATT=ATU$.



\end{frame}

\begin{frame}{What about partial independence?}

  \begin{itemize}
  \item What if selection into treatment is independent of $Y^0$ but not $Y^1$
  \item On a piece of paper, write down the independence terms and substitute it into the decomposition
  \item What is the nature of the bias?  What is eliminated?
  \end{itemize}
\end{frame}




\begin{frame}{Interference when aggregating units}

\begin{itemize}
\item While treatment effects are defined at individual level, aggregate parameters combine units
\item This therefore means that for the aggregate parameters to be stable, there cannot be ``interference'' between one unit's treatment choice and another unit's potential outcome
\item Creates challenges for definitions and estimation that are probably huge headaches, even in the RCT
\end{itemize}

\end{frame}

\begin{frame}{SUTVA}

  \begin{itemize}
    \item SUTVA stands for ``stable unit-treatment value assumption''
          \begin{enumerate}
            \item \textbf{S}: \emph{\textbf{s}table}
            \item \textbf{U}: across all \emph{\textbf{u}nits}, or the population
            \item \textbf{TV}: \emph{\textbf{t}reatment-value} (``treatment effect'', ``causal effect'')
            \item \textbf{A}: \emph{\textbf{a}ssumption}
          \end{enumerate}
    \item Largely about interference when aggregating but also poorly defined treatments and scale
  \end{itemize}
\end{frame}


\begin{frame}{SUTVA: No spillovers to other units}

  \begin{itemize}
    \item What if we impose a treatment at one neighborhood but not a contiguous one?
    \item Treatment may spill over causing $Y=Y^1$ even for the control units because of spillovers from treatment group
    \item Can be mitigated with careful delineation of treatment and control units so that interference is impossible, may even require aggregation (e.g., classroom becomes the unit, not students)
  \end{itemize}
\end{frame}



\begin{frame}{SUTVA: No Hidden Variation in Treatment}

  \begin{itemize}
    \item SUTVA requires each unit receive the same treatment dosage; this is what it means by ``stable'' (i.e., notice that the super scripts contain either 0 or 1, not 0.55, 0.27)
    \item If we are estimating the effect of aspirin on headaches, we assume treatment is 200mg per person in the treatment
    \item Easy to imagine violations if hospital quality, staffing or even the vents themselves vary across treatment group
    \item Be careful what we are and are not defining as \emph{the treatment}; you may have to think of it as multiple arms
  \end{itemize}
\end{frame}

\begin{frame}{SUTVA: Scale can affect stability of treatment effects}

  Easier to imagine this with a different example.
  \begin{itemize}
    \item Let's say we estimate a causal effect of early childhood intervention in Texas
    \item Now President Biden wants to roll it out for the whole United States -- will it have the same effect as we found?
    \item Scaling up a policy can be challenging to predict if there are rising costs of production
    \item What if expansion requires hiring lower quality teachers just to make classes?
    \item That's a general equilibrium effect; we only estimated a partial equilibrium effect (external versus internal validity)
  \end{itemize}
\end{frame}



\section{Industry example of RCT: eBay advertising}

\begin{frame}

\begin{figure}[hpt]
\begin{center}
\includegraphics[scale=0.25]{./lecture_includes/econometrica_steve.png}
\end{center}
\end{figure}

\end{frame}

\begin{frame}{Internet advertising facts}

\begin{itemize}
\item In 2012, revenues from Internet advertising was \$36.6 billion and has only grown since
\item Paid search (``search engine marketing'') is the largest format by revenue (46.3\% of 2012 revenues, or \$16.9 billion)
\item Google is leading provider (registered \$46 billion in global revenues in 2012 of which 95\% was attributed to advertising)
\end{itemize}

\end{frame}

\begin{frame}{Selection bias}

\begin{itemize}
\item Treatment was targeted ads at particular people conducting particular types of keyword search
\item Consumers who choose to click on ads are loyal and already informed about products with high likelihood to buy already 
\item Problem is ads are targeting people at the end of their search, so the question is whether they would've found it already (i.e., $E[Y^0|D=1] \neq E[Y^0|D=0]$)
\end{itemize}


\end{frame}



\begin{frame}{Selection bias}

\begin{itemize}
\item Estimated return on investment using OLS  found ROI of over 1600\%
\item Compared this to experimental methods and found ROI of -63\% with a 95\% CI of $[-124\%, -3\%]$, rejecting the hypothesis that the channel yielded short-run positive returns
\item Think back to perfect doctor -- Even without the treatment ($Y^0$), the treated group observationally would've still found a way
\end{itemize}

\end{frame}

\begin{frame}{Natural experiment}

\begin{itemize}
\item Study began with a naturally occurring and somewhat fortuitous  event at eBay
\item eBay halted SEM queries for brand words (i.e., queries that included the term eBay) on Yahoo! and Microsoft but continued to pay for these terms on Google
\item Blake, Nosky and Tadelis (2015) showed almost all of the foregone click traffic and attributed sales were captured by natural search
\item Substitution between paid and unpaid traffic was nearly one to one complete
\end{itemize}

\end{frame}


\begin{frame}

\begin{figure}
\begin{center}
\includegraphics[scale=0.2]{./lecture_includes/tadelis_fig1.png}
\end{center}
\end{figure}

\end{frame}

\begin{frame}{Interpretation of natural experiment}

\begin{quote}
``The evidence strongly supports the intuitive notion that for brand keywords, natural search is close to a perfect substitute for paid search, making brand keyword SEM ineffective for short-term sales.  After all, the users who type the brand keyword in the search query intend to reach the company's website, and most likely will execute on their intent regardless of the appearance of a paid search ad.''
\end{quote}

\end{frame}

\begin{frame}{Selection bias}

Observational data masked causal effect (recall the decomposition of the any non-designed estimation strategy)

\bigskip

\begin{quote}
``Advertising may appear to attract these consumers, when in reality they would have found other channels to visit the company's website.  We overcome this endogeneity challenge with our controlled experiments.''
\end{quote}

\end{frame}




\begin{frame}{RCT}

Natural experiment was valuable, but eBay could run a large scale RCT.

\bigskip


Use this finding of a nearly one-to-one substitution once paid search was dropped to convince eBay to field a large scale RCT discontinuing non-band key words

\bigskip


\end{frame}

\begin{frame}{Design of the experiment}

\begin{itemize}
\item Randomly assigned 30 percent of eBay's US traffic to stop all bidding for all non-brand keywords for 60 days
\item Some random group of users, in other words, were exposed to ads; a control group did not see the ads
\item Used Google's geographic bid feature that can accurately identify geographic market of the user conducting the search
\item Ads were suspended in 30 percent of markets to reduce the scope of the test and minimize the potential cost and impact to the business
\end{itemize}

\end{frame}

\begin{frame}

\begin{figure}
\begin{center}
\includegraphics[scale=0.2]{./lecture_includes/tadelis_fig3.png}
\caption{Attributed sales due to clicking on a Google link (treatment group)}
\end{center}
\end{figure}

\end{frame}


\begin{frame}

\begin{figure}
\begin{center}
\includegraphics[scale=0.2]{./lecture_includes/tadelis_fig2.png}
\caption{Differences in total sales by market (treatment to control)}
\end{center}
\end{figure}

\end{frame}

\begin{frame}

\begin{figure}
\begin{center}
\includegraphics[scale=0.2]{./lecture_includes/tadelis_ols1.png}
\caption{Spending effect on revenue using OLS but not the randomization. Effects are gigantic. }
\end{center}
\end{figure}

\end{frame}

\begin{frame}

\begin{figure}
\begin{center}
\includegraphics[scale=0.2]{./lecture_includes/tadelis_ols2.png}
\caption{Spending effect on revenue using the randomization. Effects are negative. }
\end{center}
\end{figure}

\end{frame}

\begin{frame}{Heterogenous treatment effects}

\begin{itemize}
\item Recall how the potential outcomes model explicitly models individual treatment effects could be unique and that the perfect doctor showed selection on gains masked treatment effects, perhaps even reversing sign
\item Search advertising in this RCT only worked if the consumer had no idea that the company had the desired product
\item Large firms like eBay with powerful brands will see little benefit from paid search advertising because most consumers already know that they exist, as well as what they have to offer
\end{itemize}

\end{frame}


\begin{frame}

\begin{figure}
\begin{center}
\includegraphics[scale=0.2]{./lecture_includes/tadelis_newuser_fig1.png}
\caption{Effects on new users are positive and large, but not others. }
\end{center}
\end{figure}

\end{frame}

\begin{frame}

\begin{figure}
\begin{center}
\includegraphics[scale=0.2]{./lecture_includes/tadelis_newuser_fig2.png}
\caption{Effects are largest for ``least active'' customers. }
\end{center}
\end{figure}

\end{frame}


\begin{frame}{Why are causal effects small?}

\begin{itemize}
\item They suggest that the brand query tests found small causal returns because users simply substituted from the paid search clicks to the natural search clicks
\item If that's the case, then it's explicitly a selection bias story $$E[Y^0|D=1] \neq E[Y^0|D=0]$$ where $D$ is being shown the branded advertisement based on search (i.e., they were already going there)
\item They weren't using branded search for information; they were using to \emph{navigate}
\end{itemize}

\end{frame}

\begin{frame}{Self selection based on gains}

\begin{itemize}
\item Potential outcomes is the foundation of the physical experiment because the physical experiment assigns units to treatments \emph{independent} of potential outcomes, $Y^0,Y^1$
\item This is important because outside of the physical experiment, we expect people select those important treatments based on whether, subjectively, they think $Y^1>Y^0$ or $Y^1\leq Y^0$. 
\item Rational actors almost by definition are thought to ``self-select into treatment'' making non-designed comparisons potentially misleading -- sometimes by a little, sometimes by a lot
\end{itemize}

\end{frame}


\begin{frame}{Comments}

\begin{itemize}
\item Natural experiments are valuable, but they don't always have the same certainty the way an RCT does
\item We use natural experiments when people won't let us run the RCTs we want to run!
\item Findings from natural experiments often push others to run RCTs -- like at eBay
\end{itemize}

\end{frame}


\begin{frame}{Going forward}

\begin{itemize}

\item Now let's move into a set of tools that will help us in two of the areas we cover:  DAGs
\item Matching/regression and instrumental variables both depend critically on knowing something about the data generating process
\item We'll be learning one way to assist you
\end{itemize}

\end{frame}




\end{document}